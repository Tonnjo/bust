\documentclass{article}
\usepackage[margin=1in]{geometry}
\usepackage{register}
\usepackage{enumitem}
\setlist[description]{leftmargin=\parindent,labelindent=\parindent}
\usepackage{calc}
\usepackage{tabularx}

\usepackage{listings}
\lstdefinelanguage{VHDL}{
   morekeywords=[1]{
     library,use,all,entity,is,port,in,out,end,architecture,of,
     begin,and,others
   },
   morecomment=[l]--
}
 
\lstdefinestyle{vhdl}{
   language     = VHDL,
   basicstyle   = \ttfamily,
}

\title{example{\_}module{\_}ro}
\author{}
\date{}

\begin{document}

\maketitle

\begin{description}[leftmargin=!,labelwidth=\widthof{\bfseries Address width: }]
\item [Address width: ] 32
\item [Data width: ] 32
\item [Base address: ] 0xFFAA0000
\end{description}


An example module that contain exclusively RO registers.

\section{Register List}

\begin{table}[h!]
  \begin{center}
    \label{tab:table1}
    \begin{tabularx}{\linewidth}{|l|X|l|l|l|c|l|}
      \hline
      \textbf{\#} & \textbf{Name} & \textbf{Mode} & \textbf{Address} & \textbf{Type} & \textbf{Length} &
      \textbf{Reset} \\
      \hline
      0 & reg0 & RO & \texttt{0x00000000} & SL & 1 & \texttt{0x1} \\
      \hline
      1 & reg1 & RO & \texttt{0x00000004} & SLV & 14 & \texttt{0xF} \\
      \hline
      2 & reg2 & RO & \texttt{0x00000008} & DEFAULT & 32 & \texttt{0x0} \\
      \hline
      3 & reg3 & RO & \texttt{0x0000000C} & FIELDS & 24 & \texttt{0x0} \\
      \hline
    \end{tabularx}
  \end{center}
\end{table}

\section{Registers}

\begin{register}{H}{reg0 - RO}{0x00000000}
  \par RO std{\_}logic register that resets to 0x1 for some reason \regnewline
  \label{reg0}
  \regfield{unused}{31}{1}{-}
  \regfield{}{1}{0}{1}
\reglabel{Reset}\regnewline
\end{register}

\begin{register}{H}{reg1 - RO}{0x00000004}
  \par RO std{\_}logic{\_}vector[13:0] that resets to 0xF \regnewline
  \label{reg1}
  \regfield{unused}{18}{14}{-}
  \regfield{}{14}{0}{{0xF}}
\reglabel{Reset}\regnewline
\end{register}

\begin{register}{H}{reg2 - RO}{0x00000008}
  \par Default RO register \regnewline
  \label{reg2}
  \regfield{}{32}{0}{{0x0}}
\reglabel{Reset}\regnewline
\end{register}

\begin{register}{H}{reg3 - RO}{0x0000000C}
  \par RO register with multiple types of fields \regnewline
  \label{reg3}
  \regfield{unused}{8}{24}{-}
  \regfield{field3}{3}{21}{{0x0}}
  \regfield{field2}{1}{20}{0}
  \regfield{field1}{19}{1}{{0x0}}
  \regfield{field0}{1}{0}{0}
\reglabel{Reset}\regnewline
  \begin{regdesc}\begin{reglist}[field0]
    \item [field0] std{\_}logic field    \item [field1] std{\_}logic{\_}vector[18:0] register    \item [field2] std{\_}logic field    \item [field3] std{\_}logic{\_}vector[2:0] field  \end{reglist}\end{regdesc}
\end{register}

\section{Example VHDL Register Access}

\par
All registers are bundled in records based on their mode. E.g. all RW registers are accessed through the record \textit{bustype\_rw\_regs}. Access is also dependent on the type of register. All register of type SL, SLV and DEFAULT are all directly accessed by just specifying the mode record signal. E.g. the RW register \textit{reg0} can be assigned a value like this (assuming AXI-bus):

\begin{lstlisting}[style=vhdl]
axi_rw_regs.reg0 <= (others => '0');
\end{lstlisting}

\par Registers of type FIELD cannot be directly accessed without specification of a certain field. This is because the registers are implemented as a record in VHDL (thus a record of records). E.g. if the RO register \textit{reg1} contains the field \textit{field3} it can be accessed like this (assuming AXI-bus):

\begin{lstlisting}[style=vhdl]
axi_ro_regs.reg1.field3 <= (others => '0');
\end{lstlisting}

\end{document}