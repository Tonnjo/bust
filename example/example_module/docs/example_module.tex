\documentclass{article}
\usepackage[margin=1in]{geometry}
\usepackage{register}
\usepackage{enumitem}
\setlist[description]{leftmargin=\parindent,labelindent=\parindent}
\usepackage{calc}
\usepackage{tabularx}

\usepackage{listings}
\lstdefinelanguage{VHDL}{
   morekeywords=[1]{
     library,use,all,entity,is,port,in,out,end,architecture,of,
     begin,and,others
   },
   morecomment=[l]--
}
 
\lstdefinestyle{vhdl}{
   language     = VHDL,
   basicstyle   = \ttfamily,
}

\title{example{\_}module}
\author{}
\date{}

\begin{document}

\maketitle

\begin{description}[leftmargin=!,labelwidth=\widthof{\bfseries Address width: }]
\item [Address width: ] 32
\item [Data width: ] 32
\item [Base address: ] 0xFFAA0000
\end{description}


An example module that contain all the register types that are currently supported by uart.

\section{Register List}

\begin{table}[h!]
  \begin{center}
    \label{tab:table1}
    \begin{tabularx}{\linewidth}{|l|X|l|l|l|c|l|}
      \hline
      \textbf{\#} & \textbf{Name} & \textbf{Mode} & \textbf{Address} & \textbf{Type} & \textbf{Length} &
      \textbf{Reset} \\
      \hline
      0 & reg0 & RW & \texttt{0x00000000} & SL & 1 & \texttt{0x0} \\
      \hline
      1 & reg1 & RW & \texttt{0x00000004} & SL & 1 & \texttt{0x1} \\
      \hline
      2 & reg2 & RO & \texttt{0x00000008} & SL & 1 & \texttt{0x0} \\
      \hline
      3 & reg3 & RW & \texttt{0x0000000C} & SLV & 8 & \texttt{0x3} \\
      \hline
      4 & reg4 & RO & \texttt{0x00000010} & SLV & 14 & \texttt{0x0} \\
      \hline
      5 & reg5 & RW & \texttt{0x00000014} & DEFAULT & 32 & \texttt{0xFFFFFFFF} \\
      \hline
      6 & reg6 & RO & \texttt{0x00000018} & DEFAULT & 32 & \texttt{0x0} \\
      \hline
      7 & reg7 & RW & \texttt{0x0000001C} & FIELDS & 21 & \texttt{0xAD7} \\
      \hline
      8 & reg8 & RO & \texttt{0x00000020} & FIELDS & 24 & \texttt{0x0} \\
      \hline
      9 & reg9 & PULSE & \texttt{0x00000024} & SL & 1 & \texttt{0x1} \\
      \hline
      10 & reg10 & PULSE & \texttt{0x00000028} & SLV & 4 & \texttt{0xA} \\
      \hline
      11 & reg11 & PULSE & \texttt{0x0000002C} & FIELDS & 16 & \texttt{0x3} \\
      \hline
    \end{tabularx}
  \end{center}
\end{table}

\section{Registers}

\begin{register}{H}{reg0 - RW}{0x00000000}
  \par RW std{\_}logic register that resets to 0x0 \regnewline
  \label{reg0}
  \regfield{unused}{31}{1}{-}
  \regfield{}{1}{0}{0}
\reglabel{Reset}\regnewline
\end{register}

\begin{register}{H}{reg1 - RW}{0x00000004}
  \par RW std{\_}logic register that resets to 0x1 \regnewline
  \label{reg1}
  \regfield{unused}{31}{1}{-}
  \regfield{}{1}{0}{1}
\reglabel{Reset}\regnewline
\end{register}

\begin{register}{H}{reg2 - RO}{0x00000008}
  \par RO std{\_}logic register \regnewline
  \label{reg2}
  \regfield{unused}{31}{1}{-}
  \regfield{}{1}{0}{0}
\reglabel{Reset}\regnewline
\end{register}

\begin{register}{H}{reg3 - RW}{0x0000000C}
  \par RW std{\_}logic{\_}vector[7:0] register that resets to 0x3 \regnewline
  \label{reg3}
  \regfield{unused}{24}{8}{-}
  \regfield{}{8}{0}{{0x3}}
\reglabel{Reset}\regnewline
\end{register}

\begin{register}{H}{reg4 - RO}{0x00000010}
  \par RO std{\_}logic{\_}vector[13:0] \regnewline
  \label{reg4}
  \regfield{unused}{18}{14}{-}
  \regfield{}{14}{0}{{0x0}}
\reglabel{Reset}\regnewline
\end{register}

\begin{register}{H}{reg5 - RW}{0x00000014}
  \par Default RW register that resets to 0xFFFFFFFF \regnewline
  \label{reg5}
  \regfield{}{32}{0}{{0xFFFFFFFF}}
\reglabel{Reset}\regnewline
\end{register}

\begin{register}{H}{reg6 - RO}{0x00000018}
  \par Default RO register \regnewline
  \label{reg6}
  \regfield{}{32}{0}{{0x0}}
\reglabel{Reset}\regnewline
\end{register}

\begin{register}{H}{reg7 - RW}{0x0000001C}
  \par RW register that have multiple fields \regnewline
  \label{reg7}
  \regfield{unused}{11}{21}{-}
  \regfield{field3}{15}{6}{{0x2B}}
  \regfield{field2}{1}{5}{0}
  \regfield{field1}{4}{1}{{0xB}}
  \regfield{field0}{1}{0}{1}
\reglabel{Reset}\regnewline
  \begin{regdesc}\begin{reglist}[field0]
    \item [field0] std{\_}logic that resets to 0x1    \item [field1] std{\_}logic{\_}vector[3:0] that resets to 0xb is not a valid reset value    \item [field2] std{\_}logic that resets to 0x1    \item [field3] std{\_}logic{\_}vector[14:0] that resets to 0x2b  \end{reglist}\end{regdesc}
\end{register}

\begin{register}{H}{reg8 - RO}{0x00000020}
  \par RO register with multiple types of fields \regnewline
  \label{reg8}
  \regfield{unused}{8}{24}{-}
  \regfield{field3}{3}{21}{{0x0}}
  \regfield{field2}{1}{20}{0}
  \regfield{field1}{19}{1}{{0x0}}
  \regfield{field0}{1}{0}{0}
\reglabel{Reset}\regnewline
  \begin{regdesc}\begin{reglist}[field0]
    \item [field0] std{\_}logic field    \item [field1] std{\_}logic{\_}vector[18:0] field    \item [field2] std{\_}logic field    \item [field3] std{\_}logic{\_}vector[2:0] field  \end{reglist}\end{regdesc}
\end{register}

\begin{register}{H}{reg9 - PULSE}{0x00000024}
  \par PULSE std{\_}logic register that resets to 0x1 \regnewline
  \label{reg9}
  \regfield{unused}{31}{1}{-}
  \regfield{}{1}{0}{1}
\reglabel{Reset}\regnewline
\end{register}

\begin{register}{H}{reg10 - PULSE}{0x00000028}
  \par PULSE std{\_}logic{\_}vector[3:0] register that resets to 0xA \regnewline
  \label{reg10}
  \regfield{unused}{28}{4}{-}
  \regfield{}{4}{0}{{0xA}}
\reglabel{Reset}\regnewline
\end{register}

\begin{register}{H}{reg11 - PULSE}{0x0000002C}
  \par PULSE register with two fields \regnewline
  \label{reg11}
  \regfield{unused}{16}{16}{-}
  \regfield{field1}{1}{15}{0}
  \regfield{field0}{15}{0}{{0x3}}
\reglabel{Reset}\regnewline
  \begin{regdesc}\begin{reglist}[field0]
    \item [field0] std{\_}logic{\_}vector[14:0] field that resets to 0x3    \item [field1] std{\_}logic field that resets to 0x0  \end{reglist}\end{regdesc}
\end{register}

\section{Example VHDL Register Access}

\par
All registers are bundled in records based on their mode. E.g. all RW registers are accessed through the record \textit{bustype\_rw\_regs}. Access is also dependent on the type of register. All register of type SL, SLV and DEFAULT are all directly accessed by just specifying the mode record signal. E.g. the RW register \textit{reg0} can be assigned a value like this (assuming AXI-bus):

\begin{lstlisting}[style=vhdl]
axi_rw_regs.reg0 <= (others => '0');
\end{lstlisting}

\par Registers of type FIELD cannot be directly accessed without specification of a certain field. This is because the registers are implemented as a record in VHDL (thus a record of records). E.g. if the RO register \textit{reg1} contains the field \textit{field3} it can be accessed like this (assuming AXI-bus):

\begin{lstlisting}[style=vhdl]
axi_ro_regs.reg1.field3 <= (others => '0');
\end{lstlisting}

\end{document}